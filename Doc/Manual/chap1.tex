\chapter{概要}
このマニュアルでは{\tecS}のハードウェアと,
{\tecS}に内蔵される16ビットコンピュータ{\tac}について解説を行う.

%----------------------------------
\section{\tecS}
\tecS は,
内部に{\tec}と{\tac}の二つのコンピュータを内蔵したマイコンボードである.

\begin{description}
\item[\tec]
  高校生や高専の低学年の学生が,ノイマン型コンピュータの
  動作原理を学ぶために開発された8ビットコンピュータである.
  コンソールパネルを用いて,
  二進数で機械語プログラミングを体験することができる.
  {\tec}については,
  「TeC教科書」\footnote{
    \url{https://github.com/tctsigemura/TecTextBook/raw/master/tec.pdf}}
  に詳しい説明がある.
\item[\tac]
  大学生や高専の高学年の学生が
  オペレーティングシステムやコンパイラを学習する際に,
  ターゲットとなるコンピュータのサンプルとして
  開発した16ビットコンピュータである.
  本マニュアルは,
  主に{\tac}として使用する際の{\tecS}について解説する.
\end{description}

%----------------------------------
\section{{\tecS}の外観}

\figref{TeC7Photo}に{\tecS}の写真を示す.
{\tecS}は一枚のプリント基板上に実装されている.
以下に基板の主要な部品などを紹介する.

\begin{description}
\item[コンソールパネル]
  ユーザはコンソールパネルを用いて,
  {\tec}または{\tac}のCPUレジスタやメモリの内容を読み書きしたり,
  プログラムを機械語命令単位でステップ実行したりすることができる.
  つまり,コンソールはハードウェア仕掛けのデバッガである.
  {\tec}では機械語プログラムのデバッグに,
  {\tac}ではオペレーティングシステムのデバッグに使用する.
  オペレーティングシステムのカーネル内部まで,
  ステップ実行しながらデバッグすることが可能である.
\item[JTAGコネクタ]
  FPGAを設定(コンフィグ)する設計データを書き込むために使用する.
  プリント基板上でFPGAとフラッシュメモリからなるJTAGチェインを構成しており,
  JTAGコネクタからFPGAとフラッシュメモリにアクセスすることができる.
\item[スピーカ]
  コンソールを操作した際に操作音を発生する.
  {\tec}は電子オルゴールプログラム等で使用することもできる.
\item[フラッシュメモリ]
  電源が遮断されても内容が消えないメモリである.
  {\tecS}に電源が投入された時,
  フラッシュメモリからFPGAコンフィグ用のデータが読み出される.
  内容はJTAGコネクタから書き換えることができる.
  使用している部品は,Xilinx XCF04S である.
\item[マイクロSDスロット]
  {\tac}の二次記憶装置としてマイクロSDを使用することができる.
\item[電源(USB)コネクタ]
  電源を供給するために使用する.
  また,FT232RLと接続してあるのでPCとシリアル通信をすることも可能である.
\item[FT232RL]
  PCとUSBで接続してシリアル通信をするためのICである.
\item[クロックIC]
  9.8304MHzのクロック信号を出力する水晶発振器である.
  FPGAにクロック信号を供給する.
\item[FPGA]
  {\tec},{\tac}のCPU,メモリ等,全ての主要な論理回路を内蔵する.
  使用しているFPGAは,Xilinx Spartan-6 XC6SLX9 である.
\item[RN4020]
  BLE(Bluetooth Low Energy)規格の通信モジュールである.
  FT232RLと同様な通信をBluetooth経由で行うことができる\footnote{
    通信相手には
    BlueTerminal(\url{https://github.com/tctsigemura/BlueTerminal})
    をインストールする必要がある.}.
  bバージョン以降の{\tecS}に実装されている.
\item[ジャンパ]
  {\tecS}を{\tec}として使用するか,{\tac}として使用するかを決める.
  その他に,Bluetoothモジュールのリセットや,
  デモンストレーション機能の呼び出しにも使用できる.
\end{description}

\myfigure{tbp}{width=\columnwidth}{Fig/kakubu.pdf}{TeC7の写真}{TeC7Photo}


%----------------------------------
\section{{\tecS}の内部構造}

\figref{TeC7Blk}に{\tecS}のブロック図を示す.
図中央の灰色の大きな長方形はFPGAを表しており,
主要な論理回路は全てFPGAに内蔵されていることが分かる.
FPGA内部の回路はVHDLで記述されている.
{\tecS}の回路を記述したVHDLのソースコードはGitHub\footnote{
\url{https://github.com/tctsigemura/TeC7/tree/master/VHDL}}
に公開してある.
緑色の長方形はプリント基板上に実装されたFPGA以外の部品である.
基板の回路図を付録\ref{appTac},\figref{TeC7Pcb}に示す.
以下では,\figref{TeC7Blk}を参照しながら{\tecS}の回路構成の概要を説明する.

\myfigure{tbp}{scale=.6}{Fig/TeC7.pdf}{TeC7のブロック図}{TeC7Blk}

\subsection{クロックとリセット}
DCMは,外部から供給される9.8304MHzのクロック信号から,
Spartan-6のDCM(Digital Clock Manager)を用いて,
{\tec}用の2.4576MHz,
{\tac}用の49.152MHzクロック信号を生成する.
DCMは電源投入後,クロック出力が安定すると\texttt{i\_locked}を`1'にする.

\texttt{i\_locked}が`1'になるとMODEはジャンパの設定を読み取り
結果を\texttt{i\_mode}に出力する.
ジャンパの読み取りが完了すると,
\texttt{i\_reset\_tec}と\texttt{i\_reset\_tac}が`1'になり,
{\tec}と{\tac}が動作を開始する.

\begin{mytable}{btp}{\texttt{i\_mode}の値と意味}{mode}
  \begin{tabular}{ c | l }\hline\hline
    \texttt{i\_mode} & \multicolumn{1}{|c}{意 味} \\\hline
    \texttt{000} & TeCモード({\tecS}が{\tec}として動作) \\
    \texttt{001} & TaCモード({\tecS}が{\tac}として動作) \\
    \texttt{010} & DEMO1モード(電子オルゴールプログラム入力済みのTeCモード)\\
    \texttt{011} & DEMO2モード(演奏データ入力済みのDEMO1モード)\\
    \texttt{111} & リセット(RN4020を工場出荷時の状態に戻す) \\
  \end{tabular}
\end{mytable}

\subsection{動作モード}
\label{tec7mode}
\texttt{i\_mode}の値により{\tecS}の動作モードが決まる.
\texttt{i\_mode}の値と動作モードの対応を\tabref{mode}に示す.

\begin{itemize}
\item 「TeCモード」,「DEMO1モード」,「DEMO2モード」では,
  {\tec}がコンソールと接続される.
  その様子を\figref{TeC7Blk}で確認する.
  図中の``TeC''が{\tec}コンピュータである.
  この内部に,{\tec}のCPUや主記憶,入出力装置などの回路が組み込まれている.
  \texttt{i\_mode}の値が\texttt{001}(TaCモード),
  \texttt{111}(リセットモード)以外の場合,
  図中のデマルチプレクサ(DMUX)とマルチプレクサ(MUX2)が切り替わり
  {\tec}とコンソールが接続される.
  \texttt{i\_mode}の値が「DEMOモード」の場合は,
  {\tec}のメモリに予め電子オルゴールプログラムが入力された状態になる.
\item 「TaCモード」では,{\tac}がコンソールと接続される.
  図中の``TaC''が{\tac}コンピュータである.
  \texttt{i\_mode}の値が\texttt{001}(TaCモード)の場合は,
  {\tac}にコンソールが接続される.
\item 「リセットモード」では,{\tec}も{\tac}もコンソールに接続されない.
\end{itemize}

{\tec}は「TaCモード」では停止したままになる.
一方で{\tac}は\texttt{i\_mode}の値に関係なくIPLプログラムの実行を開始する.
IPLがモードに対応した動作を行う.

\subsection{{\tac}による{\tec}の補助}
\label{tec7assist}
{\tec}のシリアル入出力(\texttt{i\_tec\_rxd/txd})は{\tac}に接続されており,
「TeCモード」では{\tac}がシリアル入出力の中継装置の役割を担う.
通常,{\tac}はシリアル入出力をFT232RLに中継する.
しかし,RN4020がBluetooth接続を確立した場合は,
FT232RLとRN4020の両方に中継するようになる.
{\tec}は{\tecS}がUSBとBluetoothのどちらで
PCに接続されているのか知る必要がない.

通常,図中の``MUX1''はコンソールを``DMUX''に接続している.
「TeCモード」時に,シリアル通信で受信した内容がTWRITEプログラム\footnote{
\texttt{Util--}(\url{https://github.com/tctsigemura/Util--})に含まれる
プログラム書き込みツールのこと.}のものなら,
通信を中継している{\tac}が``MUX1''を切り換え{\tec}のコンソールを操作し,
受信したプログラムを{\tec}のメモリに書き込む.
この機能は{\tac}のIPLプログラムに組み込まれている.

%----------------------------------
\section{{\tec}の内部構造}

\figref{TeCBlk}に{\tec}のブロック図を示す.
この図は,\figref{TeC7Blk}の``{\tec}''ブロックの内部を表している.

\myfigure{tbp}{scale=.56}{Fig/TeC.pdf}{TeCのブロック図}{TeCBlk}

\subsection{CPU,メモリ,入出力装置}
CPUとメモリや入出力装置はバスを介して接続されている.
入出力装置には,シリアル入出力(SIO),
入出力ポート(PIO),タイマー,A/Dコンバータの機能が含まれる.
SIOはTaCに接続されており,通信データはTaCがFT232RLやRN4020に中継する.
コンソールのデータSWやスピーカは入出力装置としても使用することができる.

\subsection{コンソール}
コンソールは,CPUとメモリに専用の回路で接続されている.
完全にハードウェア制御で動作するので,
プログラム実行中でも操作が可能である.

\subsection{割り込みコントローラ}
割り込みコントローラは,
コンソール,入出力装置から発生する4種類の割り込みをCPUに伝える.
CPUが割り込み認識サイクルに入ったらバスに割り込み番号を出力する.

%----------------------------------
\section{{\tac}の内部構造}

\figref{TaCBlk}に{\tac}のブロック図を示す.
この図は,\figref{TeC7Blk}の``{\tac}''ブロックの内部を表している.

\myfigure{tbp}{scale=.7}{Fig/TaC.pdf}{TaCのブロック図}{TaCBlk}

\subsection{CPU}
CPUは,バスを通してメモリや入出力と接続される.
コンソールからのStop信号が入力されている間はCPUは命令実行を停止する.
CPUがHalt機械語命令を実行した場合は,コンソールにHalt信号を出力する.

\subsection{コンソール}
コンソールのSWやLED,SPKを接続するブロックである.

コンソールは,CPUが命令実行を停止している間だけ機能する.
CPUが命令実行を開始するとコンソールの表示は変化しなくなる.
その間は,プログラムから入出力装置として使用することができる.

\subsection{割り込みコントローラ}
マイクロSDホストコントローラ,I/Oポート,タイマー,シリアルI/O,
RN4020アダプタから発生する合計10種類の割り込み信号と,
CPU,MMUから発生する6種類の例外信号を入力し,
CPUに割り込み(例外)の発生を知らせる.
CPUが割り込み(例外)認識サイクルに入ったらバスを通して
割り込み(例外)番号をCPUに伝える.

割り込み(例外)は入力信号が`0'から`1'に変化する際に発生する.
同じ種類の割り込み(例外)が複数回発生する場合は,
入力信号を一旦`0'に戻す必要がある.

\subsection{メモリ}
MMUを通してバスに接続される.
容量は64KiB(32KiW)である.
16ビット単位,または,8ビット単位のデータを読み書きできる.
16ビット単位でアクセスする場合は偶数アドレスを指定する必要がある.
マイクロSDホストコントローラやコンソールは,
バスとは別の配線でメモリに接続されおり,
DMA(Direct Memory Access)方式でメモリをアクセスすることができる.

\subsection{MMU(Memory Management Unit)}
ページング方式のMMUである.
8エントリのTLB(Translation Look-aside Buffer)を内蔵し,
8ビットのページ番号を8ビットのフレーム番号に変換する.
TLBの管理はOSが行う前提で設計されており,
ページテーブルの検索機能は持っていない.
OSはI/O命令でMMUの設定を変更できる.

\subsection{マイクロSDホストコントローラ}
マイクロSDをSPIモードに切り換え,
512バイトのセクタ単位で読み書きを行う制御をハードウェアで行う.
CPUは,LBA(Logical Block Addressing)方式で表現する32ビットのセクタアドレス,
メモリ上の512バイトのバッファのアドレスをレジスタに書き込み,
開始を指示するだけでセクタの読み書きができる.

\subsection{I/Oポート}
プリント基板上の入出力ポートと接続される.
8ビット入力,8ビット出力が基本であるが,
4ビット入力,12ビット出力に切り換えることもできる.
また,ハードウェアによるシリアル・パラレル変換機能を持つ
SPIポートとして使用することもできる.
更に,入力ポートの状態に変化があった時,
割り込みを発生する機能も持つ.

\subsection{タイマー}
1ミリ秒単位で周期を設定可能な独立した2本のインターバルタイマーである.
割り込みを発生することができる.

\subsection{シリアルI/O(SIO)}
調歩同期方式の9,600ボーのシリアル通信回路である.
プリント基板上のFT232RLと接続するもの,
{\tec}のSIOと接続するものの二つある.

\subsection{{\tec}アダプタ}
MUX1,DMUXを通して{\tec}のコンソール入力に接続されている.
このアダプタを通して{\tec}のコンソールを操作することができる.

\subsection{RN4020アダプタ}
Bluetoothモジュール(RN4020)を接続する回路である.
調歩同期方式115,200ボーのシリアル通信回路と,
RN4020の一部の外部ピンを操作・監視する回路を内蔵している.
